\documentclass[12pt]{article}
\usepackage[english]{babel}
\usepackage{natbib}
\usepackage{url}
\usepackage[utf8x]{inputenc}
\usepackage{amsmath}
\usepackage{graphicx}
\graphicspath{{images/}}
\usepackage{parskip}
\usepackage{fancyhdr}
\usepackage{vmargin}
\setmarginsrb{3 cm}{2.5 cm}{3 cm}{2.5 cm}{1 cm}{1.5 cm}{1 cm}{1.5 cm}

\title{Step-I : Mathematical Modeling}								% Title
\author{Ankit Solanki and Sumit Ghosh}								% Author
\date{\today}											% Date

\makeatletter
\let\thetitle\@title
\let\theauthor\@author
\let\thedate\@date
\makeatother

\pagestyle{fancy}
\fancyhf{}
\rhead{\theauthor}
\lhead{\thetitle}
\cfoot{\thepage}

\begin{document}

%%%%%%%%%%%%%%%%%%%%%%%%%%%%%%%%%%%%%%%%%%%%%%%%%%%%%%%%%%%%%%%%%%%%%%%%%%%%%%%%%%%%%%%%%

\begin{titlepage}
	\centering
    \vspace*{0.5 cm}
    \includegraphics[scale = 0.05]{IITD.png}\\[1.0 cm]	% University Logo
    \textsc{\LARGE Indian Institute of Technology, Delhi}\\[2.0 cm]	% University Name
	\textsc{\Large COP290}\\[0.5 cm]				% Course Code
	\textsc{\large Design Practices}\\[0.5 cm]				% Course Name
	\rule{\linewidth}{0.2 mm} \\[0.4 cm]
	{ \huge \bfseries \thetitle}\\
	\rule{\linewidth}{0.2 mm} \\[1.5 cm]
	
	\begin{minipage}{0.4\textwidth}
		\begin{flushleft} \large
			\emph{Author:}\\
			\theauthor
			\end{flushleft}
			\end{minipage}~
			\begin{minipage}{0.4\textwidth}
			\begin{flushright} \large
			\emph{Student Number:}
			2016CS50401, 2016CS50400
            % Your Student Number
		\end{flushright}
	\end{minipage}\\[2 cm]
	
	{\large \thedate}\\[2 cm]
 
	\vfill
	
\end{titlepage}

%%%%%%%%%%%%%%%%%%%%%%%%%%%%%%%%%%%%%%%%%%%%%%%%%%%%%%%%%%%%%%%%%%%%%%%%%%%%%%%%%%%%%%%%%

\tableofcontents
\pagebreak

%%%%%%%%%%%%%%%%%%%%%%%%%%%%%%%%%%%%%%%%%%%%%%%%%%%%%%%%%%%%%%%%%%%%%%%%%%%%%%%%%%%%%%%%%
\section{Part I: Flocking Behavior}
Animal behavior has always been a source of amazement to mankind. In many areas the abilities of the animals surpasses the abilities of us humans, but with the use of technology we have been able to best the animals in more and more areas. In this implementation we will try to simulate the flocking behavior seen in starling murmuration using the structure of Reynolds Boids algorithm.

\subsection{Starling Murmuration}
Starlings are small to medium-sized passerine birds in the family Sturnidae. It is known as murmuration, when a huge flocks of starling in migration form shape-shifting flight patterns.

\subsection{Flocking Behavior}
Flocking is a the motion of birds together and flocking behavior is a type of behavior exhibited when a group of birds, called a flock, are in flight.

\subsection{Boids}
Boids is the term that is commonly used to mean the computer simulation and representation of a bird in flocking simulations. The term is first floated by Craig Reynolds.

\subsection{Emergent Behavior}
When individual objects interact with each other directly or indirectly to create much more complex results. Reynolds Boids algorithm can be said to create an emergent behavior.

%----------------------------
\newpage

\section{Part II: Boid Implementaion (Pseudocode)}

The three rules acting as pillars in the realm of modern flocking simulation as presented by Craig Reynolds are - 
\newline
\newline
1. \textbf{Separation} : Steer to avoid crowding local flockmates.
\newline
2. \textbf{Alignment} : Steer towards the average heading of local flockmates.
\newline
3. \textbf{Cohesion} : Steer to move toward the average position of local flockmates.

We will here try to simulate the behavior of starlings using these 3 basic rules. 

\subsection{Separation}
Movement to avoid crowding local flock-mates. If a flocking behavior is to be simulated, it must also avoid collisions between the bird like objects. This rule attempts steer the boid away from possible collisions. Also it is important that this rule shall come into play when boids are close otherwise there will be always a resistance force when a flock is intended to form.
\newline

We will take a boid and if it's within a defined small distance of its neighboring boid then move it far away upto a certain amount. We can subtract it from a vector c the displacement of each boid which is near by.\newline
	
    function(boid B){\newline
		Vector c = 0\newline

		for each b in boids\newline
			if b != B then\newline
				if |b.position - B.position| < 100 then\newline
					c = c - (b.position - B.position)\newline
				end if\newline
			end if\newline
		end\newline
		return c\newline
	}\newline

Also, if two boids are near each other, this rule will be applied to both of them. They will be slightly steered away from each other, and at the next time step if they are still near each other they will be pushed further apart.It will be seen in the simulation in the form of a smooth acceleration and will seem closer to reality because in the actual flocking birds don't get repelled in a jerking manner, they separate in a smooth and continous manner.

\subsection{Cohesion}

Movement of boid towards the average position of local flock-mates. Cohesion as a rule will keep the flock together, without it there would not be any flocking at all.\newline

Assume we have n number of boid objects in a local-flock, namely b1, b2, ..., bn. Also, the position of a boid b is denoted b.position. Then the average position of mass in the flock shall be\newline

	c = (b1.position + b2.position + ... + bn.position) / (n-1)\newline

where positions are vector. c is also known as 'percieved center' as it is not actual center of the flock but of the local flock.\newline

	function (boid B){\newline
		Vector pc\newline
		for each b in boids\newline
			if b != B then\newline
				pc = pc + b.position\newline
			end if\newline
		end\newline
		pc = pc / n-1\newline
		return (pc - B.position) / 100\newline
}
It will move the boid 1\% closer to the center of the local group.
\newpage

\subsection{Alignment}
Movement of boid towards the average heading of local flockmates. This rule tries to make
the boids mimic each others in case of speed, acceleration and rotation(movement on axis). If this rule was not used the boids
would just get closer and not have formed the mesmerising flocking patterns that we notice and adore. \newline

In this rule, we will average the velocities. We shall introduce a 'perceived velocity' i.e. the average velocity of local group and then add a certain part of it to the boid's current velocity.

	function(boid B){ \newline

		Vector pv \newline
		int x (comment :x is the amount by which we shall increase the boid velocity) \newline

		for each b in boids \newline
			if b != B then \newline
				pv = pv + b.velocity \newline
			end if \newline
		end \newline
		pv = pv / n-1 \newline

		return (pv - B.velocity) / x \newline

	}

\textbf{Note : } These were the 3 rules originally introduced by Craig Reynolds in his computer simulation, we only need these 3 rules to implement the basic flocking simulation constantly flying. However, we shall be adding some more features to it.




% \section{Step VI: Report}
% \newpage
% * <ankit03june@gmail.com> 2018-01-16T19:00:53.688Z:
%
% ^.
% * <ankit03june@gmail.com> 2018-01-16T19:00:45.810Z:
%
% ^.
% \bibliographystyle{plain}
% \bibliography{biblist}

\end{document}